% !TeX TS-program = xelatex

\documentclass[aspectratio=169]{beamer}

\usepackage{xltxtra}
\usepackage[main=russian,english]{babel}
\defaultfontfeatures{Mapping=tex-text}
\usepackage{listings}
\usepackage[export]{adjustbox}

\usepackage[backend=biber,bibstyle=numeric,citestyle=numeric-comp,sorting=none,defernumbers=true,date=long]{biblatex}
\addbibresource{dns-security.bib}

\usepackage{ulem}

\usetheme{shl2021}

% for biblatex
\setbeamerfont{bibliography item}{size*={8pt}{1}}
\setbeamertemplate{bibliography item}{\insertbiblabel}
\renewcommand*{\bibfont}{\clbr\fontsize{8}{8}\selectfont}
\DeclareFieldFormat{url}{\color{blue}\url{#1}}

% for listings
% Solarized colors
\definecolor{sbase03}{HTML}{002B36}
\definecolor{sbase02}{HTML}{073642}
\definecolor{sbase01}{HTML}{586E75}
\definecolor{sbase00}{HTML}{657B83}
\definecolor{sbase0}{HTML}{839496}
\definecolor{sbase1}{HTML}{93A1A1}
\definecolor{sbase2}{HTML}{EEE8D5}
\definecolor{sbase3}{HTML}{FDF6E3}
\definecolor{syellow}{HTML}{B58900}
\definecolor{sorange}{HTML}{CB4B16}
\definecolor{sred}{HTML}{DC322F}
\definecolor{smagenta}{HTML}{D33682}
\definecolor{sviolet}{HTML}{6C71C4}
\definecolor{sblue}{HTML}{268BD2}
\definecolor{scyan}{HTML}{2AA198}
\definecolor{sgreen}{HTML}{859900}
\lstset{
        columns=flexible,
        keepspaces=true,
        showstringspaces=false,
        showtabs=false,
        tabsize=4,
        frame=single,
        basicstyle=\fontsize{12pt}{12}\ttfamily\color{sbase03},
        backgroundcolor=\color{sbase3},
%        keywordstyle=\color{scyan},
%        commentstyle=\color{sbase1},
%        stringstyle=\color{sblue},
%        numberstyle=\color{sviolet},
%        identifierstyle=\color{sbase00},
%        rulecolor=\color{sbase1},
        framerule=0pt
}

\lstdefinelanguage{nginx}{
        morestring=[b]{"},
        morecomment=[l]\#,
        morekeywords={map,default,server,listen,proxy_pass,ssl_preread,on}
}

% for listings background
\makeatletter
\let\old@lstKV@SwitchCases\lstKV@SwitchCases
\def\lstKV@SwitchCases#1#2#3{}
\makeatother
\usepackage{lstlinebgrd}
\makeatletter
\let\lstKV@SwitchCases\old@lstKV@SwitchCases

\lst@Key{numbers}{none}{%
    \def\lst@PlaceNumber{\lst@linebgrd}%
    \lstKV@SwitchCases{#1}%
    {none:\\%
     left:\def\lst@PlaceNumber{\llap{\normalfont
                \lst@numberstyle{\thelstnumber}\kern\lst@numbersep}\lst@linebgrd}\\%
     right:\def\lst@PlaceNumber{\rlap{\normalfont
                \kern\linewidth \kern\lst@numbersep
                \lst@numberstyle{\thelstnumber}}\lst@linebgrd}%
    }{\PackageError{Listings}{Numbers #1 unknown}\@ehc}}
\makeatother

\makeatletter
%%%%%%%%%%%%%%%%%%%%%%%%%%%%%%%%%%%%%%%%%%%%%%%%%%%%%%%%%%%%%%%%%%%%%%%%%%%%%%
%
% \btIfInRange{number}{range list}{TRUE}{FALSE}
%
% Test in int number <number> is element of a (comma separated) list of ranges
% (such as: {1,3-5,7,10-12,14}) and processes <TRUE> or <FALSE> respectively

\newcount\bt@rangea
\newcount\bt@rangeb

\newcommand\btIfInRange[2]{%
    \global\let\bt@inrange\@secondoftwo%
    \edef\bt@rangelist{#2}%
    \foreach \range in \bt@rangelist {%
        \afterassignment\bt@getrangeb%
        \bt@rangea=0\range\relax%
        \pgfmathtruncatemacro\result{ ( #1 >= \bt@rangea) && (#1 <= \bt@rangeb) }%
        \ifnum\result=1\relax%
            \breakforeach%
            \global\let\bt@inrange\@firstoftwo%
        \fi%
    }%
    \bt@inrange%
}
\newcommand\bt@getrangeb{%
    \@ifnextchar\relax%
        {\bt@rangeb=\bt@rangea}%
        {\@getrangeb}%
}
\def\@getrangeb-#1\relax{%
    \ifx\relax#1\relax%
        \bt@rangeb=100000%   \maxdimen is too large for pgfmath
    \else%
        \bt@rangeb=#1\relax%
    \fi%
}

%%%%%%%%%%%%%%%%%%%%%%%%%%%%%%%%%%%%%%%%%%%%%%%%%%%%%%%%%%%%%%%%%%%%%%%%%%%%%%
%
% \btLstHL<overlay spec>{range list}
%
% TODO BUG: \btLstHL commands can not yet be accumulated if more than one overlay spec match.
% 
\newcommand<>{\btLstHL}[1]{%
\only#2{\btIfInRange{\value{lstnumber}}{#1}{\color{sbase2}\def\lst@linebgrdcmd{\color@block}}{\def\lst@linebgrdcmd####1####2####3{}}}%
}%
\makeatother

\title{Безопасность DNS}
\author{Филипп Кулин}

\begin{document}

\begin{frame}
\titlepage
\end{frame}

\begin{frame}[fragile]{Откиньтесь на спинку кресла}
        \begin{columns}[T,onlytextwidth]
                \begin{column}{0.56\textwidth}
                        Knot 3.0+
                        \begin{lstlisting}
ppa:cz.nic-labs/knot-dns-latest
knot-dnsutils

copr @cznic/knot-dns-latest
knot-utils \end{lstlisting}
                \end{column}
                \begin{column}{0.40\textwidth}
                        ISC BIND 9.17.11+
                        \begin{lstlisting}
ppa:isc/bind-dev
bind9-dnsutils

copr isc/bind-dev
isc-bind-bind-utils\end{lstlisting}
                \end{column}
        \end{columns}
        \begin{lstlisting}
docker cznic/knot:latest \end{lstlisting}
        \begin{itemize}
                \item Эта презентация сделана с помощью \LaTeX
                % я 20 лет занимался хостингом и DNS был моим хобби.
                % мы в разработке довольно мало уделяем внимание деталям работы DNS (и это нормально). 
                % я считаю, что это немного легкомысленным.
                % я расскажу о имеющихся способах защиты и их применения
                \item Я расскажу страшную сказку про DNS
        \end{itemize}
\end{frame}

\begin{frame}{DNS --- всему голова}
        \begin{itemize}
                \item Жизнь пользователей в сети
                \item Запросы к API, работа с CDN
                \item Облака, микросервисы, автообнаружение и конфигурация
                \item Невообразимое количество всего
        \end{itemize}
\end{frame}

\begin{frame}{Тайная жизнь привычных программ}
% Мои любимые примеры
        \begin{itemize}
                \item<1-> SSHD определяет домен для подключившегося IP\\{\small и этот факт является одним из источников седых волос у админов}
                \item<2-> MySQL определяет домен для подключившегося IP
                \item<3-> Apache определяет домен для подключившегося IP\\{\small даже если \texttt{HostnameLookups Off}, но есть \texttt{Require}}
                \item<4-> Microsoft Windows постоянно шлёт DNS Update в сеть
                \item<5-> Docker, Kubernetes, etc...
                \item<6-> \textbf{Запустите tcpdump/WireShark}
        \end{itemize}
\end{frame}

\begin{frame}{DNS --- это просто?}
        Три каверзных вопроса:
        \begin{itemize}
                \item Каков максимальный размер доменного имени? % Паскалевские строки с длиной в начале и 63 символами, последняя длина 0, всё вместе 255
                \item Точку на конце надо ставить? % 1. Это удобно программам 2. Их этого произошли правила интерпретации
                \item Что именно спрашивает ресолвер, и что отвечают DNS-сервера при рекурсивном обходе?
        \end{itemize}
        % во время подготовки к докладу я нашел ошибку/неточность в сервере NSD
\end{frame}

\begin{frame}[t]{Как устроен DNS}%
        \includegraphics[height=0.8\textheight]{pic1.png}
\end{frame}

\begin{frame}{Особенности классического DNS}
        \begin{itemize}
                \item UDP-транспорт. Нет соединения
                \item Нет идентификации серверов DNS
                \item Нет контроля данных
                \item Нет шифрования
                \item \textbf{Мало что изменилось с тех пор}
        \end{itemize}
\end{frame}

\begin{frame}{Угорозы в системе DNS}
        \centering\includegraphics[height=0.75\textheight]{pic2.png}
\end{frame}

\begin{frame}{Заложенная в DNS безопасность}
        \onslide<2->{``... действия, которые с современной точки зрения могут показаться неправильными или ошибочными,
        часто оказывались естественным следствием господствовавшего в те времена понимания тех или иных вещей, а
        также ограниченности доступных ресурсов.``}
        \onslide<2-> \rightline{{\rm --- Брайан Керниган}}\supercite{unix-memoir}
\end{frame}

% завидую хакерам 80-ых и 90-ых - халява

% https://en.wikipedia.org/wiki/DNS_hijacking
% A number of consumer ISPs such as AT&T,[4] Cablevision's Optimum Online,[5] CenturyLink,[6] Cox Communications, RCN,[7] Rogers,[8] Charter Communications (Spectrum), Plusnet,[9] Verizon,[10] Sprint,[11] T-Mobile US,[12] Virgin Media,[13][14] Frontier Communications, Bell Sympatico,[15] Deutsche Telekom AG,[16] Optus,[17] Mediacom,[18] ONO,[19] TalkTalk,[20] Bigpond (Telstra),[21][22][23][24] TTNET, Türksat, and Telkom Indonesia[25] use or used DNS hijacking for their own purposes, such as displaying advertisements[26] or collecting statistics. Dutch ISPs XS4ALL and Ziggo use DNS hijacking by court order: they were ordered to block access to The Pirate Bay and display a warning page instead.[27] These practices violate the RFC standard for DNS (NXDOMAIN) responses,[28] and can potentially open users to cross-site scripting attacks.[26]

% офигенно
% https://www.internetsociety.org/blog/2018/04/amazons-route-53-bgp-hijack/


% https://www.trendmicro.com/vinfo/us/security/news/cyber-attacks/hacked-or-spoofed-digging-into-the-malaysia-airlines-website-compromise

\begin{frame}{Основные проблемы}
        \begin{itemize}
                \item Подделка
                \item Прослушка
        \end{itemize}
\end{frame}

\begin{frame}[fragile]{Основные проблемы. Подделка}
        \begin{itemize}
                \item Отравление, подмена
                \item Взлом серверов и замена записей
                \item Поддельные серверы, BGP-injection
                \begin{itemize}
                        \item Атака на Route53 в апреле 2018 года \supercite{route53-bgp-hijacking} \\ % https://www.internetsociety.org/blog/2018/04/amazons-route-53-bgp-hijack/
                        {\small \textcolor{blue}{\href{https://www.internetsociety.org/blog/2018/04/amazons-route-53-bgp-hijack/}{www.internetsociety.org/blog/2018/04/amazons-route-53-bgp-hijack/}}}
                \end{itemize}
                \item Госрегулирование
                \begin{itemize}
                        \item Блокировка сайтов в Европе и России
                        \begin{lstlisting}
dig +short @a.res-nsdi.ru. rutracker.org A \end{lstlisting}
                \end{itemize}
        \end{itemize}
\end{frame}

\begin{frame}{Основные проблемы. Прослушка}
        \begin{itemize}
                \item<1-> Реклама, сбор статистики, что-то ещё \supercite{dns-hijacking}\\
                {\small \textcolor{blue}{\href{https://en.wikipedia.org/wiki/DNS\_hijacking\#Manipulation\_by\_ISPs}{en.wikipedia.org/wiki/DNS\_hijacking\#Manipulation\_by\_ISPs}}}
                \item<1-> Шпионаж и промышленный шпионаж
                \begin{itemize}
                        \item<2-> ... с использованием госрегулирования
                \end{itemize}
                \item<1-> RFC7626: 73.1\% могут быть узнаны по слепку DNS \supercite{rfc-7816}
                \item<2-> Госрегулирование
                \begin{itemize}
                        \item<2-> Помощь в оперативной блокировке \supercite{mt-free-pre-block}\\
                        {\small \textcolor{blue}{\href{https://usher2.club/articles/mt-free-pre-block/}{usher2.club/articles/mt-free-pre-block/}}}
                \end{itemize}
        \end{itemize}
\end{frame}

\begin{frame}{А так ли страшен чёрт?}
        \centering\vskip0.5cm\includegraphics[height=0.7\textheight]{monster.jpg}<1>
        \begin{itemize}
                \item<2-> Вы знаете, кто, когда и как использует какой DNS?
                \item<3-> Ваш сетевой периметр защищен? Точно?
                \item<4-> Ваша сеть получает подписанные маршруты?
                        \begin{itemize}
                                \item<5-> Вы ведете журнал странных анонсов?
                        \end{itemize}
                \item<6-> Ваши сервисы проверяют сертификат соединения? % при обращении наружу, журнал этого есть?
                \item<7-> \textbf{Однако, современные взломы чаще основаны на бардаке}
        \end{itemize}
\end{frame}

\begin{frame}{Защита от подделки}
        \begin{itemize}
                \item Не «взлетевший» DNSCurve
                \item Расширение DNSSEC
        \end{itemize}
\end{frame}

\begin{frame}{DNSCurve}
                Концепция
                \begin{itemize}
                        \item Аутентификация авторитативного DNS-сервера
                        \item Защита обмена между ресолвером и авторитативным сервером
                \end{itemize}
                Принцип действия
                \begin{itemize}
                        \item Публичный ключ DNS-сервера с магическим префиксом \texttt{"uz5"} в NS-записи домена:\\
                                {\small\texttt{\textbf{uz5}qry75vfy162c239jgx7v2knkwb01g3d04qd4379s6mtcx2f0828.dnscurve.io}}
                        \item Обмен с DNS-сервером шифруется
                \end{itemize}
\end{frame}

\begin{frame}{DNSCurve. Особенности}
        \begin{itemize}
                \item Не меняет саму спецификацию DNS
                \item Основан на вере в целостность системы
                \item Не предусмотрена замена ключа
                \item \textbf{Зависит от источника ответа}
                \item Внедрение практически отсутствует
        \end{itemize}
\end{frame}

\begin{frame}{DNSSEC}
        \begin{itemize}
                \item Концепция
                \begin{itemize}
                        \item Источник записи неважен. Используя доверенный корневой ключ, возможно проверить любую подписанную запись
                \end{itemize}
                \item Принцип действия
                \begin{itemize}
                        \item Записи зоны подписаны ключом зоны %важно - не сервера
                        \item Подтверждения подписи выстраиваются в цепочку доверия
                \end{itemize}
        \end{itemize}
\end{frame}

\begin{frame}{DNSSEC. Принцип действия}
        \begin{columns}[T,onlytextwidth]
                \begin{column}{0.5\textwidth}<1->
                        \textbf{Подпись зоны}
                        \includegraphics[height=0.7\textheight]{pic3.png}
                \end{column}
                \begin{column}{0.5\textwidth}<2->
                        \textbf{Цепочка доверия}
                        \includegraphics[height=0.7\textheight]{pic4.png}
                \end{column}
        \end{columns}
\end{frame}

\begin{frame}{DNSSEC. Особенности}
        \begin{itemize}
                \item \textbf{Источник ответа неважен}
                \item Требует аккуратности и непрерывного обслуживания даже в статическом состоянии
                \item Требует стартовых настроек клиента \\
                {\small требуются актуальные корневые ключи}
                \item Сложные реализации «отрицательного ответа»
                \item Большой размер ответа
                \item Крайне слабая глубина внедрения
                \item \textbf{Это единственный вариант в этой категории} % нет никаких других защит от подделки
        \end{itemize}
\end{frame}

\begin{frame}[fragile]{DNSSEC. Настройка клиентов} % выбор в зависимости от доверия и цели
        \begin{itemize}
                \item Прозрачная проверка\\
                {\small Потребитель получает фильтрованные ответы } % основной вариант использования
                \item Явная проверка\\
                {\small Потребитель явно указывает ресолверу, что хочет получить проверенный результат. Проверяет флаги ответа }
                \item Усиленная проверка\\
                {\small Потребитель проверяет подписи сам } % для использования расширений это основной вариант
        \end{itemize}
        \begin{lstlisting}
delv @8.8.8.8 dxdt.ru A\end{lstlisting}
\end{frame}

\begin{frame}{DNSSEC. Must have}
        \begin{itemize}
                \item<1-> Подпишите свои домены
                \begin{itemize}
                        \item<1-> CoreDNS и Knot DNS --- отличные реализации
                \end{itemize}
                \item<2-> Настройте ваши ресолверы на проверку DNSSEC
                \begin{itemize}
                        \item<2-> CoreDNS не умеет проверять DNSSEC
                        \item<2-> \texttt{\sout{systemd-resolved}, unbound, Knot Resolver} --- умеют
                \end{itemize}
        \end{itemize}
\end{frame}

\begin{frame}{DNSSEC. Вкусняшка SSHFP}
        SSH Fingerprint
        \begin{itemize}
                \item Запись SSHFP содержит хэш публичного ключа хоста
                \item На клиенте \emph{~/.ssh/config}: \texttt{VerifyHostKeyDNS yes}
                \item На сервере \texttt{ssh-keygen -R `hostname`}
                \begin{itemize}
                        \item \small{Не надо все алгоритмы, не тяните за собой легаси}
                \end{itemize}
                \item Работает только с DNSSEC
                \item RFC 4255 --- SSH Fingerprint\supercite{rfc-4255}
        \end{itemize}
\end{frame}

% DoT autodiscover https://knot-resolver.readthedocs.io/en/stable/modules-experimental_dot_auth.html
% Великобритания https://www.opennet.ru/opennews/art.shtml?num=51046

\begin{frame}{Защита от прослушки DNS}
        \onslide<1-> Шифрование сообщений
        \begin{itemize}
                \item<1->DNSCrypt
        \end{itemize}
        \onslide<2-> Защищенный канал
        \begin{itemize}
                \item<2->DNS-over-HTTPS Google API
                \item<2->DNS-over-TLS
                \item<2->DNS-over-HTTP/2
                \item<2->DNS-over-QUIC
        \end{itemize}
        \onslide<3-> Прочее
        \begin{itemize}
                \item<3->Минимизация QNAME при запросах
                \item<3->EDNS0 Client subnets
        \end{itemize}
\end{frame}

\begin{frame}[fragile]{DNSCrypt}
        Принцип действия
        \begin{itemize}
                \item Настройка мастер-ключа и имени сервера
                \item Получение «короткого» ключа и сертификата
                \item Запросы к серверу, идентичные DNSCurve
        \end{itemize}
\begin{lstlisting}[basicstyle=\fontsize{10pt}{10}\ttfamily\color{sbase03}]
dig @77.88.8.78 -p 15353 2.dnscrypt-cert.browser.yandex.net. \
        -t TXT +short\end{lstlisting}
\end{frame}

\begin{frame}{DNSCrypt. Особенности}
        \begin{itemize}
                \item Не меняет спецификацию DNS
                \item Нет ни RFC, ни Draft. Только спецификация на сайте
                \item Не предусмотрена замена мастер-ключа
                \item Заметное количество программ
                \item \textbf{Нет автообнаружения}
                \item Не «взлетел»
        \end{itemize}
\end{frame}

\begin{frame}[fragile]{DNS-over-HTTPS (Google API)}
        Google предоставляет JSON-API к DNS\\
        Страница с описанием:\\
        \small{\textcolor{blue}{\href{https://developers.google.com/speed/public-dns/docs/dns-over-https}{https://developers.google.com/speed/public-dns/docs/dns-over-https}}}
        ~\\
        Массово используется для веб-приложений
        \begin{lstlisting}[basicstyle=\fontsize{10pt}{10}\ttfamily\color{sbase03}]
curl -H 'accept: application/dns-json' \
  'https://dns.google/resolve?name=example.com' | jq

curl -H 'accept: application/dns-json' \
  'https://cloudflare-dns.com/dns-query?name=example.com' | jq\end{lstlisting}
\end{frame}


\begin{frame}[fragile]{DNS-over-TLS (DoT)}
        \begin{itemize}
                \item<1-> Устанавливается защищенное TLS-соединение (порт 853)
                \item<1-> Внутри соединения – стандартный DNS-протокол
                \item<1-> {\small Самая простая инсталляция – проксирование \texttt{nginx} через \texttt{ngx\_stream\_ssl\_module} на обычный DNS}
        \end{itemize}
        \begin{lstlisting}
kdig +tls @8.8.8.8 highload.ru # попробуйте 195.208.4.1

dig +tls @1.1.1.1 highload.ru\end{lstlisting}
        \begin{itemize}
                \item<2-> А есть ещё DNS-over-DTLS...
                \item<3-> ... и DNS-over-QUIC...
        \end{itemize}
\end{frame}

\begin{frame}{DNS-over-TLS (DoT). Особенности}
        \begin{itemize}
                \item Не меняет спецификацию DNS
                \item Требует установки TLS-соединения (дорого)
                \item Требует стартовых настроек клиента \\
                {\small требует «бутстрапа» имени сервера} % Не бывает сертификатов на IP (если вы не Cloudflare)
                \item \textbf{Нет автообнаружения}
                \item Специальный 853 порт
        \end{itemize}
\end{frame}

\begin{frame}[fragile]{DNS-over-HTTPS (DoH)}
        \begin{itemize}
                \item Защищенным транспортом является обычный HTTP/2
                \item Запросы/ответы --- стандартные DNS-пакеты
                \item Формируется специальный HTTP-запрос
                \begin{itemize}
                        \item GET --- DNS-пакет кодируется в параметр
                        \item POST --- DNS-пакет в \texttt{application/dns-message}
                \end{itemize}
        \end{itemize}
        \begin{lstlisting}
kdig +https @8.8.8.8 highload.ru # попробуйте 195.208.4.1

dig +https @1.1.1.1 highload.ru\end{lstlisting}
\end{frame}

\begin{frame}{DNS-over-HTTPS (DoH). Особенности}
        \begin{itemize}
                \item Не меняет спецификацию DNS
                \item Требует установки HTTP/2-соединения (дорого)
                \item Требует стартовых настроек клиента \\
                {\small требует «бутстрапа» имени сервера} % Не бывает сертификатов на IP (если вы не Cloudflare)
                \item \textbf{Нет автообнаружения}
                \item \textbf{Не сильно выделяется в HTTP-трафике}
        \end{itemize}
\end{frame}

% Added in 7.62.0
% curl --doh-url https://cloudflare-dns.com/dns-query https://www.google.com

% dig BIND 9.17.11 25 ноября 2020

% curl -H 'accept: application/dns-message' 'https://dns.google/dns-query?dns=q80BAAABAAAAAAAAA3d3dwdleGFtcGxlA2NvbQAAAQAB'  | hexdump -c

% knot 3.0.0, September 9, 2020
% kdig @1.1.1.1 +https example.com.
% kdig @193.17.47.1 +https=/doh example.com.
% kdig @8.8.4.4 +https +https-get example.com.
% kdig @8.8.8.8 +https +tls-hostname=dns.google +fastopen example.com.

\begin{frame}{Защита. Must have}
        \begin{itemize}
                \item Локальные кэши в каждом периметре
                \begin{itemize}
                        \item NodeLocal DNSCache в Kubernetes
                        \item \sout{systemd-resolved}, unbound, Knot Resolver
                \end{itemize}
                \item DoT/DoH через недоверенные сети \\
                {\small особенно локальные домены}
                \item Агрегирующие DoT/DoH-ресолверы
        \end{itemize}
\end{frame}

\begin{frame}{Защита канала. Вот незадача}
        \centering\includegraphics[height=0.7\textheight]{pic6.png}
\end{frame}

% Experimental DNS-over-TLS Auto-discovery
% https://github.com/CZ-NIC/knot-resolver/tree/c8cb9740f8ebd34219c7d860106969fcbb6c7bf6/modules/experimental_dot_auth

% https://labs.ripe.net/media/images/fig2_2wBmhvi.width-800.png
\begin{frame}{Минимизация QNAME}
        \centering\includegraphics[height=0.7\textheight]{qname.png}
\end{frame}

\begin{frame}[fragile]{EDNS Client subnet}
        Это расширение DNS
        \begin{itemize}
                \item<1-> Добавляет в запрос подсеть клиента
                \item<1-> Например, для геобалансинга
        \end{itemize}
        Поддержка
        \begin{itemize}
                \item<1-> Google DNS принципиально \textbf{да}
                \item[]<2-> \begin{lstlisting}
dig +short @8.8.8.8 -t TXT o-o.myaddr.l.google.com\end{lstlisting}%
                \item<1-> Cloudflare DNS принципиально \textbf{нет}
                \item[]<2-> \begin{lstlisting}
dig +short @1.1.1.1 -t TXT o-o.myaddr.l.google.com\end{lstlisting}%
        \end{itemize}
\end{frame}

\begin{frame}{Известные сервисы отладки}
	\begin{itemize}
		\item \texttt{whoami.akamai.net A}
		\item \texttt{whoami.akamai.net AAAA}
		\item \texttt{o-o.myaddr.l.google.com TXT}
		\item \texttt{whoami.cloudflare.com TXT}
		\item \texttt{whoami.ipv6.akahelp.net TXT}
		\item \texttt{whoami.ipv4.akahelp.net TXT}
		\item \texttt{whoami.ds.akahelp.net TXT}
	\end{itemize}
\end{frame}

\begin{frame}[fragile]{Как проверить ресолвер}
        Google Public DNS. DNS blocking and hijacking\supercite{google-ts}
        \begin{lstlisting}
dig -t TXT test.dns.google.com. '@dns.google.'

dig -t TXT +tcp locations.dns.google.com. '@dns.google.'\end{lstlisting}
        Have problems with 1.1.1.1? *Read Me First*\supercite{cloudflare-ts}
        \begin{lstlisting}
dig +short CHAOS TXT id.server @1.1.1.1

dig @1.1.1.1 whoami.Cloudflare.com txt +short\end{lstlisting}
\end{frame}

\begin{frame}[fragile]{Версия сервера}
        \begin{lstlisting}
dig +short -c CHAOS -t TXT version.bind @8.8.8.8

dig +short -c CHAOS -t TXT id.server @1.1.1.1\end{lstlisting}%
        \begin{itemize}
                \item RFC4892 --- идентификация сервера\supercite{rfc-4892}
                \item \texttt{HOSTNAME.BIND, VERSION.BIND, ID.SERVER}
        \end{itemize}
\end{frame}

\begin{frame}{Настройка локальных кэшей}
        \begin{itemize}
                \item Включение/выключение QNAME
                \item Манипуляции с Client subnet
        \end{itemize}
\end{frame}

\begin{frame}{Реакционизм. Подделка} % претензии к защите
        \begin{itemize}
                \item Не позволяет подставлять «свой» ответ
                \begin{itemize}
                        \item Противоречит корпоративным политикам
                        \item Мешает спецслужбам проводить спецоперации
                \end{itemize}
                \item Переусложненное обслуживание приводит к ошибкам
        \end{itemize}
\end{frame}

\begin{frame}{Реакционизм. Прослушка} % претензии к защите
        %\centering\vskip0.5cm\includegraphics[height=0.7\textheight]{rodina.jpg}<1>
        \begin{itemize}
                \item<1-> Не позволяет анализировать DNS-запросы
                \begin{itemize}
                        \item<1-> Нарушает корпоративные стандарты безопасности
                        \item<1-> Мешает приложениям защиты отслеживать действия браузера
                        \item<1-> Создаёт видимость безопасности % конечно, есть серебряная пуля
                \end{itemize}
                \item<2-> Дополнительная нагрузка
                \item<2-> Цикл получения ответа неприемлемо долгий
        \end{itemize}
\end{frame}

\begin{frame}{Реакционизм. Гоcрегулирование} % претензии к защите 
        \begin{itemize}
                \item Давление UK ISPA и IWF \\ % https://www.opennet.ru/opennews/art.shtml?num=51046
                \small{\textcolor{blue}{\href{https://www.opennet.ru/opennews/art.shtml?num=51046}{www.opennet.ru/opennews/art.shtml?num=51046}}}
                \item Большинство «госблокировок» в мире основано на манипуляциях с DNS
        \end{itemize}
\end{frame}

\begin{frame}{Национальная система доменных имен}
         \includegraphics[height=0.75\textheight,center]{pic5.png}
\end{frame}

\begin{frame}{Госрегулирование РФ. НСДИ}
        Национальная система доменных имен
        \begin{itemize}
                \item Определена в законе 90-ФЗ от 01.05.2019 \\
                {\small Приказ Роскомнадзора от 31.07.2019 № 229} % http://publication.pravo.gov.ru/Document/View/0001201911080052
                \item Государственный публичный DNS
                \item Дублирует \textbf{.} (корень)
                \item Уменьшает ущерб от манипуляций с \textbf{\texttt{.RU}} \\
                {\small гипотетических, со стороны США в лице ICANN}
                \item Обслуживается ЦМУ ССОП
                \item Предоставляется в том числе AXFR
        \end{itemize}
\end{frame}

\begin{frame}{Заложенная в НСДИ безопасность}
         \includegraphics[height=0.5\textheight,center]{travolta.png}<2->
\end{frame}

\begin{frame}{Держитесь подальше\\от торфяных болот}
        \begin{itemize}
                \item<1-> DNS таит в себе множество угроз
                \item<1-> Не используйте DNS для всего на свете
                \item<1-> Избегайте использования DNS
                \item[]<2->{\small когда есть специализированные сервисы}
        \end{itemize}
\end{frame}

\begin{frame}{Многое осталось за кадром}
        \begin{itemize}
                \item EDNS(0) Padding, Cookies, etc...
                \item Обслуживание DNS, DNSSEC, DoT/DoH
                \item Применение DNSSEC: DANE, etc...
                \item Обзор серверов, включая \texttt{stub-ресолверы}
                \item Обзор клиентов и инструментов
                \item DNS Stamps (ссылки sdna://)
                \item \texttt{glibc} и \texttt{resolv.conf}
                \item Ampliphication attack, etc...
                \item \textbf{...}
        \end{itemize}
\end{frame}

\begin{frame}{Вопросы}
        \vskip2cm
        \center В любом случае пишите мне
        \vskip2cm
        \center{schors@gmail.com}
\end{frame}

%%% Всякая хрень
% https://dnscrypt.info/stamps-specifications  ссылки sdns://
% https://github.com/ameshkov/dnscrypt клиент dnscrypt


\nocite{*}
\setbeamertemplate{frametitle continuation}{}

\begin{frame}[t]{Ссылки. DNSCurve и DNSCrypt}
\printbibliography[keyword={dnscurve},notkeyword={en}]
\printbibliography[keyword={dnscrypt},notkeyword={en}]
\end{frame}

\begin{frame}[t]{Ссылки. DNSSEC}
\printbibliography[keyword={dnssec},notkeyword={en}]
\end{frame}

\begin{frame}[t]{Ссылки. DoH/Dot}
\printbibliography[keyword={doh},notkeyword={en}]
\end{frame}

\begin{frame}[t]{Ссылки. Инциденты}
\printbibliography[keyword={incidents},notkeyword={en}]
\end{frame}

\begin{frame}[t]{Ссылки. Разное}
\printbibliography[keyword={extra},notkeyword={en}]
\end{frame}

\begin{frame}[t]{Ссылки. \LaTeX }
\printbibliography[keyword={latex},notkeyword={en}]
\end{frame}

\end{document}


